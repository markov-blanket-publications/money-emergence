\documentclass{article}

% ready for submission
\usepackage{arxiv}

% to compile a camera-ready version, add the [final] option, e.g.:
% \usepackage[final]{neurips_2018}

\usepackage[utf8]{inputenc} % allow utf-8 input
\usepackage[T1]{fontenc}    % use 8-bit T1 fonts
\usepackage{hyperref}       % hyperlinks
\usepackage{url}            % simple URL typesetting
\usepackage{booktabs}       % professional-quality tables
\usepackage{amsfonts}       % blackboard math symbols
\usepackage{nicefrac}       % compact symbols for 1/2, etc.
\usepackage{microtype}      % microtypography
\usepackage{graphicx}

\title{The emergence of money as a topological phase transition}

\date{January 6, 2021}


% The \author macro works with any number of authors. There are two commands
% used to separate the names and addresses of multiple authors: \And and \AND.
%
% Using \And between authors leaves it to LaTeX to determine where to break the
% lines. Using \AND forces a line break at that point. So, if LaTeX puts 3 of 4
% authors names on the first line, and the last on the second line, try using
% \AND instead of \And before the third author name.

\author{%
  Aidan Rocke\\
  \texttt{aidanrocke@gmail.com} \\
  % examples of more authors
  % \And
  % Coauthor \\
  % Affiliation \\
  % Address \\
  % \texttt{email} \\
  % \AND
  % Coauthor \\
  % Affiliation \\
  % Address \\
  % \texttt{email} \\
  % \And
  % Coauthor \\
  % Affiliation \\
  % Address \\
  % \texttt{email} \\
  % \And
  % Coauthor \\
  % Affiliation \\
  % Address \\
  % \texttt{email} \\
}

\begin{document}
% \nipsfinalcopy is no longer used

\maketitle

\begin{abstract}
   Through an open-ended thought experiment, I explore how money 
   emerges through the minimisation of network costs that grow quadratically in a barter network. Furthermore, I explore how a monetary system allows the complementary developments of commercial technology which makes civilised life possible, and a Leviathan without which civilisation would be impossible. 
\end{abstract}

\section{Barter economies in pre-industrialised societies}

In a hunter-gatherer society, we may suppose that $N$ goods and commodities are produced
and that these are represented as vertices $\{v_i\}_{i=1}^N$ in a weighted undirected graph
that represents a barter network. This means that the graph is fully-connected and an
accountant in such a society would need a table of size:

\begin{equation}
f(N) = \\{N \choose 2 \\} = \frac{N \cdot (N-1)}{2} \sim N^2
\end{equation}

in order to keep track of the relative values of different goods/commodities $v_i$. These
relative values may be represented as rational numbers on each edge, the weight of each
edge if you will.

This is manageable for small $N$ but if the creative power of the society increases significantly due to either population growth or the discovery of metallurgy then the administrative cost of managing $\sim N^2$ exchanges becomes intractable.

One hypothetical solution would be to create a central bank that issues a standard currency
which is convertible to an existing commodity(ex. iron), denoted by $v_1$. If
this currency consists of coin made of a particular iron alloy, then the growth rate of the
amount of money in circulation is proportional to the rate of discovery of new iron ores, and
therefore a proxy measure for economic growth.

This new currency, which we may denote by $v_{N+1}$ may now be added as a good to the existing
barter network. In particular, it is unique in simultaneously serving the following functions:

(1) Medium of exchange: a device used to facilitate the exchange of goods between parties

(2) Store of value: given that the coin is durable and convertible to its weight in iron alloy, its shelf life is practically indefinite

(3) Measure of value: we now have a numerical system for comparing the value of different goods/commodities

Given these relatively unique functions and the fact that the value of this currency is proportional to the value of $v_1$ in mass, the relative value(i.e. prices) of all the other nodes may be defined in terms of $v_{N+1}$. From a graph-theoretical perspective, we delete
all the existing edges while adding a vertex $v_{N+1}$ which we connect to all the other
vertices. Given that a connected graph with $N+1$ vertices must have at least $N$ edges,
our solution is both unique and optimal. Moreover, given that the change in graph topology coincides with a qualitative change in the behaviour of the underlying physical system(i.e. economic behaviour)
we may say that a topological phase transition has occurred.

\newpage

\section{The transition to industrial societies with monetary systems}

Thanks to the creation of money, the administrative cost is now proportional to $N$ rather than
$N^2$. But, the gains are much more important than this. In fact, a monumental phase-transition
has occurred.

Given that a monetary system establishes a standard measure of value, an accounting system may
now be established. Business entities may now be easily created and taxed annually. Furthermore,
loans may be issued by the central bank to a network of smaller banks which may in turn issue
loans to small businesses in order to stimulate business development.

The stable development of these industrial institutions as well as the security of growing
trade routes would not be assured without the enforcement of the rule of law. So a monetary
system effectively requires the establishment of a Leviathan which may create several institutions
to assure homeland security. In these circumstances, a nation-state is born.

Given that parallel developments are likely to take place in neighboring regions, homeland
security will necessarily include a well-equipped military and a well-funded department of
defense that works on countermeasures for the destructive potential of new dual-use technologies.
In addition, it would be wise to fund the development of intelligence agencies without which
foreign diplomacy would be impossible. The importance of foreign diplomacy may not be under-estimated as this institution significantly reduces the risk of unnecessary conflict, and
facilitates international cooperation on shared risks and opportunities such as
counter-terrorism and nuclear disarmament.

This Leviathan owes its legitimacy partly to its ability to execute economic development plans
through public-private partnerships that serve its citizens. In modern times, this may include
the development of public transportation infrastructure, public education systems, water
treatment and waste management systems, telecommunication systems and energy grids. The
control of such technologies and understanding their interactions is not a simple task, but
these must be handled by a government on a day to day basis. Crucially, these are all
technologies that will improve its citizens' well-being as well as improve the productivity of
labour which shall sustain economic growth.

However, an equally important part of this Leviathan's legitimacy is owed to a Justice
system founded upon a moral system with metaphysical origins. In particular, this moral
system depends upon resolute answers to the following questions:

1.  Why do we exist?

2. What makes us unique?

So the Leviathan's legitimacy effectively depends upon metaphysical questions of existence
and uniqueness. On the US dollar bill, both these questions are answered by the phrase, 'In God we trust.'. As an important corollary, the day a nation begins to doubt in the legitimacy of its moral system
marks the beginning of its decline and fall.


\section{Challenges}
While the creation of monetary systems allows the development and evolution of a powerful
nation-state, power is also insufficient to justify authority. This creates a problem of trust on an unprecedented scale whose only solution is the population's shared belief in the legitimacy of a moral system with metaphysical origins. Yet, if unwavering belief is necessary it is also insufficient as the growth and evolution
of a nation-state implies the emergence of complex systems which requires an increased number
of specialised engineering skills. The increasing interaction of a number of different technologies means that we also need an increased
number of versatile generalists to manage technologies(ex. biotech) that inter-operate across different industries. 

Without a sufficient number of generalists systemic risks may grow unchecked. This is particularly important in the
development of new technologies, new systems of government such as smart cities, and financial
institutions that are especially vulnerable to the madness of crowds. Now, given that the history of civilisation is a testament to humanity's ability to believe in all
kinds of myths and the exponential increase in civilisation's complexity due to technological
developments is a relatively recent phenomenon, I believe important reforms to the education
system are necessary if an increasingly complex civilisation is not to crumble under the
weight of its sheer complexity. 

In a world of increasingly big datasets it is important to never forget that big data never speaks for itself and the limits of our language represent the limits of our world. 

\section*{References}

\small

[1] Thomas Hobbes. Leviathan: Or the Matter, Forme and Power of a Commonwealth, Ecclesiasticall and Civil. 1651.

[2] Nick Szabo. Shelling Out: The Origins of Money. 2002.

[3] Ben Green. The Smart Enough City. The MIT Press. 2019.

[4] René Girard. Violence and the Sacred. John Hopkins University Press. 1979.

[5] Miklós Bóna. A Walk Through Combinatorics: An Introduction To Enumeration And Graph Theory. World Scientific. 2002.   

\end{document}